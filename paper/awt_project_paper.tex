\documentclass[conference]{IEEEtran}
\IEEEoverridecommandlockouts
% The preceding line is only needed to identify funding in the first footnote. If that is unneeded, please comment it out.
\usepackage{cite}
\usepackage{amsmath,amssymb,amsfonts}
\usepackage{algorithmic}
\usepackage{graphicx}
\usepackage{textcomp}
\usepackage{xcolor}
\def\BibTeX{{\rm B\kern-.05em{\sc i\kern-.025em b}\kern-.08em
    T\kern-.1667em\lower.7ex\hbox{E}\kern-.125emX}}

\begin{document}

\title{User Interface for Video and Image Stitching}

\author{
\IEEEauthorblockN{1\textsuperscript{st} Nguyen, Trung Han}
\IEEEauthorblockA{\textit{Open Distributed Systems} \\
\textit{Technische Universität Berlin}\\
Berlin, Germnay \\
trung.h.nguyen@campus.tu-berlin.de
}
\and
\IEEEauthorblockN{2\textsuperscript{nd} Tran, Minh Duc}
\IEEEauthorblockA{\textit{Open Distributed Systems} \\
\textit{Technische Universität Berlin}\\
Berlin, Germnay \\
minh.d.tran@campus.tu-berlin.de
}
\and
\IEEEauthorblockN{3\textsuperscript{rd} Tran, Nhat Duc}
\IEEEauthorblockA{\textit{Open Distributed Systems} \\
\textit{Technische Universität Berlin}\\
Berlin, Germnay \\
nhat.d.tran@campus.tu-berlin.de
}
}

\maketitle

\begin{abstract}
This document is a model and instructions for \LaTeX.
This and the IEEEtran.cls file define the components of your paper [title, text, heads, etc.]. *CRITICAL: Do Not Use Symbols, Special Characters, Footnotes, 
or Math in Paper Title or Abstract.
\end{abstract}

\begin{IEEEkeywords}
web UI, video and image stitching, .mpd, SQS
\end{IEEEkeywords}

\section{Introduction}
\textbf{ToDo: Nate} \\
- Fraunhofer FOKUS is part of the Fraunhofer society and is engaged in applied research and development in the field of Information and Communications Technology \\
- one of the works include the development of a backend for video and image stitching \\
- similar to a typical video editing program like windows movie maker but via web \\
- idea: put together video and image urls and in the end return a stitched video .mpd file url \\
- AWS' SQS serves as interface for communication \\
- as of now: missing a proper UI for seamless interaction with the FH backend \\
- in this work we implemented a web UI for that very purpose \\
- agenda: section II. looks into related works and what makes this work special.
sections III. and IV. deals with the architecture and implementation. here non-trivial steps are described and important decisions made while developing the UI are explained.
section V. evaluates the approach taken and discusses its strong points but also its flaws.
section VI. concludes this work. suggestions for future work are proposed

\section{Related Work}
\textbf{ToDo: Han} \\
List online/offline video stitching tools, differences and what makes our web app unique...

\section{Architecture \& Used Technologies}
\textbf{ToDo: Han} \\
Web UI communicates with FH backend via SQS; using AngularJS, AWS SDK, clouds...

\section{Design \& Implementation}
List some features etc. but this section will only focus on main aspects of our web ui implementation and why things were done that way...

\subsection{General}
UI primarily for Desktop, save session via .txt export...

\subsection{Content}
\textbf{ToDo: Minh} \\
URL input for adding new contents from web/clouds...

\subsection{Timeline Area}
\textbf{ToDo: Minh} \\
most controls using hammer JS...

\subsection{Preview Player}
\textbf{ToDo: Nate} \\
z-index for better loading, playback-loop (parallel playing media while counting time) ...

\subsection{SQS}
\textbf{ToDo: Nate} \\
\textit{Segmentation Request}.
set timeout approach while polling only 1 message at a time...

\textit{Stitching Request}.
same approach as segmentation, doesn't work due to backend...

\section{Discussion}
SQS and FH backend are shit; a backend managing the request and providing an API would be better; suitability for use with regard to user-friendliness, performance, robustness...

\section{Conclusion}
webUI great, SQS and FH backend are shit...

\textit{Future Work}. open features and improvements...

\begin{thebibliography}{00}
\bibitem{b1} G. Eason, B. Noble, and I. N. Sneddon, ``On certain integrals of Lipschitz-Hankel type involving products of Bessel functions,'' Phil. Trans. Roy. Soc. London, vol. A247, pp. 529--551, April 1955.
\bibitem{b2} J. Clerk Maxwell, A Treatise on Electricity and Magnetism, 3rd ed., vol. 2. Oxford: Clarendon, 1892, pp.68--73.
\bibitem{b3} I. S. Jacobs and C. P. Bean, ``Fine particles, thin films and exchange anisotropy,'' in Magnetism, vol. III, G. T. Rado and H. Suhl, Eds. New York: Academic, 1963, pp. 271--350.
\bibitem{b4} K. Elissa, ``Title of paper if known,'' unpublished.
\bibitem{b5} R. Nicole, ``Title of paper with only first word capitalized,'' J. Name Stand. Abbrev., in press.
\bibitem{b6} Y. Yorozu, M. Hirano, K. Oka, and Y. Tagawa, ``Electron spectroscopy studies on magneto-optical media and plastic substrate interface,'' IEEE Transl. J. Magn. Japan, vol. 2, pp. 740--741, August 1987 [Digests 9th Annual Conf. Magnetics Japan, p. 301, 1982].
\bibitem{b7} M. Young, The Technical Writer's Handbook. Mill Valley, CA: University Science, 1989.
\end{thebibliography}

\end{document}
