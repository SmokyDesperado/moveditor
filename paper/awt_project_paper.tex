\documentclass[conference]{IEEEtran}
\IEEEoverridecommandlockouts
% The preceding line is only needed to identify funding in the first footnote. If that is unneeded, please comment it out.
\usepackage{cite}
\usepackage{amsmath,amssymb,amsfonts}
\usepackage{algorithmic}
\usepackage{graphicx}
\usepackage{textcomp}
\usepackage{xcolor}
\def\BibTeX{{\rm B\kern-.05em{\sc i\kern-.025em b}\kern-.08em
    T\kern-.1667em\lower.7ex\hbox{E}\kern-.125emX}}

\begin{document}

\title{User Interface for Video and Image Stitching}

\author{
\IEEEauthorblockN{1\textsuperscript{st} Nguyen, Trung Han}
\IEEEauthorblockA{\textit{Open Distributed Systems} \\
\textit{Technische Universität Berlin}\\
Berlin, Germnay \\
trung.h.nguyen@campus.tu-berlin.de
}
\and
\IEEEauthorblockN{2\textsuperscript{nd} Tran, Minh Duc}
\IEEEauthorblockA{\textit{Open Distributed Systems} \\
\textit{Technische Universität Berlin}\\
Berlin, Germnay \\
minh.d.tran@campus.tu-berlin.de
}
\and
\IEEEauthorblockN{3\textsuperscript{rd} Tran, Nhat Duc}
\IEEEauthorblockA{\textit{Open Distributed Systems} \\
\textit{Technische Universität Berlin}\\
Berlin, Germnay \\
nhat.d.tran@campus.tu-berlin.de
}
}

\maketitle

\begin{abstract}
This document is a model and instructions for \LaTeX.
This and the IEEEtran.cls file define the components of your paper [title, text, heads, etc.]. *CRITICAL: Do Not Use Symbols, Special Characters, Footnotes, 
or Math in Paper Title or Abstract.
\end{abstract}

\begin{IEEEkeywords}
web UI, video and image stitching, MPD, DASH, Amazon Simple Queue Service (SQS)
\end{IEEEkeywords}

\section{Introduction}
With the recent progresses in cloud computing in the areas of cost-efficiency, scalability etc., more and more software applications are shifted from local computers to external servers.
While in the past software like audio or graphics editor tools had to be installed locally due to high processing demands, nowadays mentioned applications can be provided over the internet.

The Fraunhofer FOKUS \textit{(Fraunhofer Institute for Open Communication Systems)} is a research unit of the Fraunhofer society and is concerned with applied research and development of applications in the field of Information and Communications Technology.
Among the projects of FOKUS is the development of such a service for video and image stitching, similar to well-known traditional software like Window's Movie Maker, but in this case over the internet.

Using media URLs from any host or cloud provider, the backend creates an MPD file (manifest file for adaptive media streaming) according to a given stitching configuration.
Amazon's Simple Queue Service (SQS) serves as communication channel between frontend and service backend.

As of now FOKUS has only implemented a stitching backend -- a proper user interface for seamless interaction with the backend is still missing.
In this work we implemented a web UI\footnote{Code available at https://github.com/SmokyDesperado/moveditor} for that very purpose.

In particular this inlcudes:
\textbf{(II)} A glimpse into related work and what elevates this work.
\textbf{(III)} and \textbf{(IV)} The Architecture and implementation, in which non-trivial steps are described and important decisions made while developing the UI are explained.
\textbf{(V)} An evaluation of the approach taken for the implemented user interface as well as a discussion of its strong points but also its flaws.
\textbf{(VI)} A conclusion to this work and suggestions for future.

\section{Related Work}
\textbf{ToDo: Han} \\
List online/offline video stitching tools, differences and what makes our web app unique...

\section{Architecture \& Used Technologies}
\textbf{ToDo: Han} \\
Web UI communicates with FH backend via SQS; using AngularJS, AWS SDK, clouds...

\section{Design \& Implementation}
List some features etc. but this section will only focus on main aspects of our web ui implementation and why things were done that way...

\subsection{General}
UI primarily for Desktop, save session via .txt export...

\subsection{Content}
\textbf{ToDo: Minh} \\
URL input for adding new contents from web/clouds...

\subsection{Timeline Area}
\textbf{ToDo: Minh} \\
most controls using hammer JS...

\subsection{Preview Player}
\textbf{ToDo: Nate} \\
z-index for better loading, playback-loop (parallel playing media while counting time) ...

\subsection{SQS}
\textbf{ToDo: Nate} \\
\textit{Segmentation Request}.
set timeout approach while polling only 1 message at a time...

\textit{Stitching Request}.
same approach as segmentation, doesn't work due to backend...

\section{Discussion}
SQS and FH backend are shit; a backend managing the request and providing an API would be better; suitability for use with regard to user-friendliness, performance, robustness...

\section{Conclusion}
webUI great, SQS and FH backend are shit...

\textit{Future Work}. open features and improvements...

\begin{thebibliography}{00}
\bibitem{b1} G. Eason, B. Noble, and I. N. Sneddon, ``On certain integrals of Lipschitz-Hankel type involving products of Bessel functions,'' Phil. Trans. Roy. Soc. London, vol. A247, pp. 529--551, April 1955.
\bibitem{b2} J. Clerk Maxwell, A Treatise on Electricity and Magnetism, 3rd ed., vol. 2. Oxford: Clarendon, 1892, pp.68--73.
\bibitem{b3} I. S. Jacobs and C. P. Bean, ``Fine particles, thin films and exchange anisotropy,'' in Magnetism, vol. III, G. T. Rado and H. Suhl, Eds. New York: Academic, 1963, pp. 271--350.
\bibitem{b4} K. Elissa, ``Title of paper if known,'' unpublished.
\bibitem{b5} R. Nicole, ``Title of paper with only first word capitalized,'' J. Name Stand. Abbrev., in press.
\bibitem{b6} Y. Yorozu, M. Hirano, K. Oka, and Y. Tagawa, ``Electron spectroscopy studies on magneto-optical media and plastic substrate interface,'' IEEE Transl. J. Magn. Japan, vol. 2, pp. 740--741, August 1987 [Digests 9th Annual Conf. Magnetics Japan, p. 301, 1982].
\bibitem{b7} M. Young, The Technical Writer's Handbook. Mill Valley, CA: University Science, 1989.
\end{thebibliography}

\end{document}
